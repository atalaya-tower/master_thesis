\documentclass[xetex, aspectratio = 169]{beamer}
\usecolortheme{beaver}
\beamertemplatenavigationsymbolsempty
\setbeamertemplate{footline}[frame number]

\usepackage{fontspec}
\usepackage{xecyr}
\usepackage[russian]{babel}

\usepackage{fvextra}
\usepackage[babel]{microtype}
\usepackage[autostyle]{csquotes}

\defaultfontfeatures{Ligatures = {TeX, Historic}, Mapping = tex-text}
\setmainfont{CMU Serif}
\setsansfont{CMU Sans Serif}
\setmonofont{CMU Typewriter Text}
%\newfontfamily{\agio}{AgioUnicode}

\usepackage{booktabs}
\usepackage{multirow}
\usepackage{makecell}
\usepackage{tabularx}
\renewcommand\theadfont{\normalsize}

\usepackage{xcolor}
\usepackage{textcomp}
\usepackage{graphicx}
\usepackage{tikz-er2}
\graphicspath{{fig/}}

\usepackage{schemata}
\usepackage{fancyvrb}

\usepackage[group-separator={,}]{siunitx}
\usepackage[russian]{datetime2}

\begin{document}

\title{Градуировка тональной шкалы реакций \\ на~интернет-новости \\ (на материале русского, английского и испанского языков)}
\author{Савинцева Дарья Викторовна}
\subtitle{Образовательная программа ВМ.5626 \\ "<Прикладная и экспериментальная лингвистика"> \\ Профиль "<Компьютерная лингвистика и~интеллектуальные технологии">}
\institute{Научный руководитель: к.",ф.",н., доц.\ Азарова~И.",В.}
\date{\DTMdisplaydate{2020}{06}{16}{5}}

\frame{\titlepage}

%\begin{frame}{Цель и задачи работы}

%	Цель: создать тональную шкалу тем, представленных в комментариях пользователей социальных сетей на актуальные новости.
%	\vspace{0.5em}

%	Задачи:
%	\begin{enumerate}
%		\item<1-> Определить рабочую модель тонального анализа на базе современных методов автоматического анализа текстов:
%		\begin{itemize}
%			\item Подобрать и дополнить существующие тональные лексиконы для русского, английского и испанского языков;
%			\item Определить модель тематического моделирования;
%		\end{itemize}
%		\item<2-> Создать корпусы комментариев в новостных группах социальных сетей на русском, английском и испанском языках;
%		\item<3-> Разработать процедуру выделения тональных компонентов в текстах, используя тематическое моделирование и тональные лексиконы;
%		\item<4-> Разработать шкалу для сопоставления тональной окраски тем.
%	\end{enumerate}
%\end{frame}

\begin{frame}{Цель и задачи работы}
    \begin{block}{Цель}
        Создать тональную шкалу тем, представленных в комментариях пользователей социальных сетей на актуальные новости.
    \end{block}

    \begin{block}{Задачи}
        \begin{enumerate}
            \item Определить рабочую модель тонального анализа на базе современных методов автоматического анализа текстов:
            \begin{itemize}
                \item Подобрать и дополнить существующие тональные лексиконы для русского, английского и испанского языков;
                \item Определить модель тематического моделирования;
            \end{itemize}
            \item Создать корпусы комментариев в новостных группах социальных сетей на русском, английском и испанском языках;
            \item Разработать процедуру выделения тональных компонентов в текстах, используя тематическое моделирование и тональные лексиконы;
            \item Разработать шкалу для сопоставления тональной окраски тем.
        \end{enumerate}
    \end{block}
\end{frame}

\begin{frame}{Анализ тональности и тематическое моделирование}
    \begin{block}{Анализ тональности}
        \begin{itemize}
            \item Это набор методов выделения эмоционально-экспрессивно окрашенных компонентов в~тексте.
            \item В исследовании используется словарный подход.
        \end{itemize}
    \end{block}

    \begin{block}{Тематическое моделирование}
        \begin{itemize}
            \item Это инструмент статистического анализа текста, задача которого выделить темы в~коллекции документов, то, как эти темы связаны друг с~другом.
            \item В исследовании используется алгоритм тематического моделирования латентное размещение Дирихле, реализованный в библиотеке gensim.
        \end{itemize}
    \end{block}
\end{frame}

\begin{frame}{Гипотеза}

	Распределение тональных компонентов в темах комментариев может показать:
	\vspace{0.5em}

	\begin{itemize}
		\item насколько тонально окрашена та или иная тема в сравнении с остальными;
		\item какова средняя тональная окраска темы;
		\item разницу в тональной окраске тем в разных языках.
	\end{itemize}
	\vspace{0.5em}

	Так станет возможно расположить темы на тональной шкале, основанной на тональности комментариев пользователей.

\end{frame}

\begin{frame}{Корпусы комментариев}
    \begin{table}
        \small
        \centering
        \begin{tabularx}{\textwidth}{XXXXX}
            \toprule
            \centering \multirow{4}{*}{\thead{Язык}} &
            \centering \multirow{4}{*}{\thead{Источники \\ (кол-во)}} &
            \centering \multirow{4}{*}{\thead{Комментарии}} &
            \multicolumn{2}{c}{\thead{Словарь}} \\ \cmidrule{4-5}
            & & & \thead{до \\ обработки} & \thead{после \\ обработки} \\ \midrule\midrule
            русский & Вконтакте (47) & \num{6022501} & \num{2003523} & \num{557091} \\ \midrule
            английский & YouTube (15) & \num{6463610} & \num{1158766} & \num{399738} \\ \midrule
            испанский & YouTube (15) & \num{526436} & \num{284714} & \num{101858} \\
            \bottomrule
        \end{tabularx}
    \end{table}
\end{frame}

\begin{frame}[fragile]
	\frametitle{Тональный словарь для русского языка}
	\begin{columns}
		\column{0.5\linewidth}
		\begin{itemize}
            \item создан на основе словаря Linis-crowd
            \item \num{11653} единицы
            \item \(-1\) или \(1\) тональность
        \end{itemize}
        \column{0.5\linewidth}
        \begin{Verbatim}[fontsize=\small]
        подвиг,Nn,1
        подвижник,Nn,1
        подвижница,Nn,1
        подвижнический,Aj,1
        подвижничество,Nn,1
        подвижность,Nn,1
        подвижный,Aj,1
        подводить,Vb,-1
        подворовывание,Nn,-1
        подворовывать,Vb,-1
        подвох,Nn,-1
        \end{Verbatim}
	\end{columns}
\end{frame}

\begin{frame}[fragile]
    \frametitle{Тональный словарь для английского языка}
    \begin{columns}
        \column{0.5\linewidth}
        \begin{itemize}
            \item Opinion (Sentiment) Lexicon
            \item \num{6789} единиц
            \item \(-1\) или \(1\) тональность
        \end{itemize}
        \column{0.5\linewidth}
        \begin{Verbatim}[fontsize=\small]
        outsider,-1
        outsmart,1
        outstanding,1
        outstandingly,1
        outstrip,1
        outwit,1
        ovation,1
        over-acted,-1
        over-awe,-1
        over-balanced,-1
        over-hyped,-1
        \end{Verbatim}
    \end{columns}
\end{frame}

\begin{frame}[fragile]
    \frametitle{Тональный словарь для испанского языка}
    \begin{columns}
        \column{0.5\linewidth}
        \begin{itemize}
            \item создан на основе словарей improved Spanish Opinion Lexicon и Spanish Sentiment Lexicon
            \item \num{10463} единицы
            \item \(-1\) или \(1\) тональность
        \end{itemize}
        \column{0.5\linewidth}
        \begin{Verbatim}[fontsize=\small]
        tirana,-1
        tiranas,-1
        tirania,-1
        tiranica,-1
        tiranicamente,-1
        tiranicas,-1
        tiranico,-1
        tiranicos,-1
        tiranizar,-1
        tirano,-1
        tiranos,-1
        \end{Verbatim}
    \end{columns}
\end{frame}

\begin{frame}{Выделенные темы в корпусах}
    \small
    \centering
    \begin{tabular}{l|l|l} \hline
        {Русский} & {Американский} & {Латиноамериканский} \\ \hline\hline
        общественные вопросы & общественное мнение & демократия \\
        ситуация в стране & ложь и страх & венесуэла и россия \\
        бытовые проблемы & здравоохранение & революционные настроения \\
        споры & коронавирус & правительство \\
        нужды человека & бог & воровство и терроризм \\
        путин и страна & трамп~и ситуация в~стране & президент \\
        работа & человеческие чувства & проблемы людей \\
        карантин & расизм & освобождение \\
        финансовые проблемы & законопорядок & ложь и предательство \\
        семья & деньги и работа & ожидания \\
        жизнь в россии & ненависть и обман & ситуация в мире \\
        коронавирус в мире & повседневная жизнь & базовые нужды \\
        удалённые или спам & америка, патриотизм & сша и латинская америка \\
        повседневная жизнь & выборы & коммунисты и~диктатура \\
        образование & криминал и коррупция & трамп \\
        \hline
    \end{tabular}
\end{frame}

\begin{frame}{Визуализация полученных тем}
	\centering
	\includegraphics[width=0.6\linewidth]{tp_meduza_rus}

	\includegraphics[width=0.6\linewidth]{tp_huffpost_usa}
\end{frame}

\begin{frame}{Распределение тональных компонентов в российском корпусе}
	\begin{columns}
		\column{0.3\linewidth}
		\centering
		\includegraphics[width=0.75\linewidth]{total_sentim_comp_rus}
		\vspace{0.3em}

		Положительные и отрицательные

		\column{0.3\linewidth}
		\centering
		\includegraphics[width=0.75\linewidth]{neg_sentim_rus}
		\vspace{0.3em}

		Отрицательные

		\column{0.3\linewidth}
		\centering
		\includegraphics[width=0.75\linewidth]{pos_sentim_rus}
		\vspace{0.3em}

		Положительные
	\end{columns}
\end{frame}

\begin{frame}{Распределение тональных компонентов в американском корпусе}
    \begin{columns}
        \column{0.3\linewidth}
        \centering
        \includegraphics[width=0.75\linewidth]{all_sentims_usa}
        \vspace{0.3em}

        Положительные и отрицательные

        \column{0.3\linewidth}
        \centering
        \includegraphics[width=0.75\linewidth]{neg_sentims_usa}
        \vspace{0.3em}

        Отрицательные

        \column{0.3\linewidth}
        \centering
        \includegraphics[width=0.75\linewidth]{pos_sentims_usa}
        \vspace{0.3em}

        Положительные
    \end{columns}
\end{frame}

\begin{frame}{Распределение тональных компонентов в латиноамериканском корпусе}
    \begin{columns}
        \column{0.3\linewidth}
        \centering
        \includegraphics[width=0.75\linewidth]{total_sentim_comp_al}
        \vspace{0.3em}

        Положительные и отрицательные

        \column{0.3\linewidth}
        \centering
        \includegraphics[width=0.75\linewidth]{negative_sentim_al}
        \vspace{0.3em}

        Отрицательные

        \column{0.3\linewidth}
        \centering
        \includegraphics[width=0.75\linewidth]{positive_sentim_al}
        \vspace{0.3em}

        Положительные
    \end{columns}
\end{frame}

\begin{frame}{Среднее значение тональности для каждой темы}
    \begin{columns}
        \column{0.3\linewidth}
        \centering
        \includegraphics[width=0.75\linewidth]{avg_sentim_rus}
        \vspace{0.3em}

        Российский корпус

        \column{0.3\linewidth}
        \centering
        \includegraphics[width=0.75\linewidth]{avg_sent_usa}
        \vspace{0.3em}

        Американский корпус

        \column{0.3\linewidth}
        \centering
        \includegraphics[width=0.75\linewidth]{avg_sentim_al}
        \vspace{0.3em}

        Латиноамериканский корпус
    \end{columns}
\end{frame}

\begin{frame}{Тональная шкала тем}
    \begin{center}
        \begin{tabular}{lll}
            {Тема} & {Сумма средних} & {Среднее суммы} \\ \hline\hline
            {бытовые проблемы} & \num{-0.31} & \num{-0.15} \\
            {нужды человека} & \num{-0.30} & \num{-0.10} \\
            {ситуация в стране} & \num{-0.29} & \num{-0.10} \\
            {общественные вопросы} & \num{-0.26} & \num{-0.09} \\
            {коронавирус} & \num{-0.23} & \num{-0.11} \\
            {президент} & \num{-0.20} & \num{-0.10} \\
            {повседневная жизнь} & \num{-0.19} & \num{-0.01} \\
            {ситуация в мире} & \num{-0.14} & \num{-0.07} \\
        \end{tabular}
    \end{center}

\end{frame}

\begin{frame}{Выводы}
    \small
	\begin{itemize}
		\item Созданы корпусы комментариев к новостным группам на русском, английском и испанском языке с помошью материалов российских, американских и латиноамериканских СМИ;
		\vspace{0.5em}

		\item С помощью тематического моделирования выделены темы, присутствующие в комментариях, построены тематические профили сообществ;
		\vspace{0.5em}

		\item Темы размечены при помощи тональных словарей, показаны распределения тональных компонентов в темах;
		\vspace{0.5em}

		\item Определено среднее значение тональности для каждой темы, на основании которого построена тональная шкала тем для сходных тем из трёх корпусов;

        \item Вывод исследования: внимание пользователей в комментариях новостных сообществ и каналов всех исследуемых групп сосредоточено больше на общественно-политических темах, нежели чем на личных и бытовых, но бытовые и общественные проблемы находят у пользователей больший эмоциональный отклик, чем темы политики и ситуации в мире.
	\end{itemize}
\end{frame}

\end{document}